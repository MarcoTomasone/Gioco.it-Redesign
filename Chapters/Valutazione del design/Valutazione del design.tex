\documentclass[../Report.tex]{subfiles}
\usepackage[italian]{babel}
\graphicspath{ {../../Images/} }

\begin{document}
    \chapter{Valutazione del Design}
    In questo capitolo andremo a valutare quello che è il design ottenuto dalle fasi precedenti. La valutazione sarà formata da due step:
    \begin{itemize}
        \item \textbf{L'inspezione:} nella quale andremo a valutare noi stessi in qualità di membri del team il design ottenuto. Useremo la tecnica del \emph{cognitive walkthrougth}, una tecnica che sfrutta l'approccio drammaturgico narrativo.
        \item  \textbf{Il testing:} nella quale verra chiesto ad alcuni utenti di utilizzare i nostri wireframe per valutare l'usabilità del sito. 
    \end{itemize}

    \section{Inspezione}
    Per quanto riguarda l'inspezione stileremo dapprima una lista di task che vogliamo analizzare.
    \begin{enumerate}
        % SIMONE I PRIMI TRE - LUCA GLI ULTIMI 3
        \item Accedere al profilo
        \item Giocare al gioco "Subway surf", metterlo nei preferiti e lasciare un commento. Eliminare poi il commento.
        \item Cercare un utente e aggiungerlo alla lista degli amici.
        \item Modificare la propria foto profilo, caricandone una personalizzata.
        \item Eliminare un gioco dai preferiti.
        \item Impostare il parental control con un limite di due ore di utilizzo ed un blocco per i giochi che superino i 16 anni.
        \item Aggiungere al parental control il blocco per i giochi violenti.
        \item Disabilitare il parental control.
        \item Creare un nuovo account.
    \end{enumerate}
    \subsection{Cognitive Walkthrought}
    \subsubsection{Cognitive walkthrough 7:}
    \textbf{Task:} Aggiungere al parental control il blocco per i giochi violenti.\\
    \textbf{User:} Un padre di 45 anni come Claudio (vedere \ref{section: personas}).\\

    \subsubsection{Happy path}
    \begin{enumerate}
        \item Cliccare sul pulsante con l’omino in alto a destra sulla HomePage
        \item Dal menu che si apre scegliere Parental Control
        \item Inserire la password per accedere al Parental Control (000000)
        \item Nella nuova schermata che si presenta, cliccare sul pulsante "+" al box "Vuoi limitare l'utilizzo di alcuni giochi?"
        \item Dalla finestra di dialogo cercare la categoria violenza (se non è già presente)
        \item Premere il pulsante "+" di fianco la categoria "Violenza"
    \end{enumerate}

    \subsubsection{Walkthrough:}
    Claudio è sulla HomePage del sito Giochi.it e comincia ad esplorarlo. Individua in alto a destra il pulsante del profilo e ci clicca sopra e a dal menu che viene fuori clicca sulla sezione Parental Control. Successivamente, all'apertura della nuova pagina, individua l'input text da inserire e inserisce inizialmente una password errata, poi si corregge e inserisce 000000, quella giusta. Dopo essersi aperta la schermata del Parental Control, Claudio, individua il box di limite dei giochi, clicca sul pulsante "+" e nella finestra modale che compare individua subito la categoria Violenza e la seleziona con il pulsante "+".  
    \subsubsection{Valutazione:}


    \subsubsection{Miglioramenti:}
    Inserire un messaggio di errore quando la password inserita è sbagliata.

    \subsubsection{Cognitive walkthrough 8:}
    \textbf{Task:} Disabilitare il Parental Control.\\
    \textbf{User:} Una ragazza di 35 anni come Silvia (vedere \ref{section: personas}).\\

    \subsubsection{Happy path}
    \begin{enumerate}
        \item Cliccare sul pulsante con l’omino in alto a destra sulla HomePage
        \item Dal menu che si apre cliccare sul nome del profilo
        \item Dalla nuova schermata che si apre cliccare sul pulsante con lo scudo, di fianco l'immagine del profilo
        \item Inserire la password per accedere al Parental Control (000000)
        \item Nella nuova schermata che si presenta, cliccare sul bottone radio "No", alla domanda "Vuoi attivare il parental control?"
    \end{enumerate}

    \subsubsection{Walkthrough:}
    Silvia vuola disattivare il Parental Control, usato per i suoi alunni, per giocare. Inizialmente è sulla HomePage del sito Giochi.it e comincia trova in alto a destra il pulsante del profilo e ci clicca sopra e dal menu che viene fuori clicca sulla sezione del suo profilo. Successivamente, all'apertura della nuova pagina, individua il pulsante con lo scudo del Parental Control e ci clicca sopra e inserisce la password in modo corretto nella text box per accedervi. Dopo essersi aperta la schermata del Parental Control, Silvia, individua il box di limite dei giochi, clicca sul pulsante "+" e nella finestra modale che compare individua subito la categoria Violenza e la seleziona con il pulsante "+".  
    \subsubsection{Valutazione:}


    \subsubsection{Miglioramenti:}

    \subsection{Valutazione Euristiche (opzionale)}
    \section{User Testing}
    \subsection{Protocollo di testing}
    \subsection{Testing}
    \subsection{Analisi dei risultati}

    \chapter{Conclusioni}
\end{document}