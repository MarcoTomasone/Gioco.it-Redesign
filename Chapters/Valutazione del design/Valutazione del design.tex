\documentclass[../Report.tex]{subfiles}
\usepackage[italian]{babel}
\graphicspath{ {../../Images/} }

\begin{document}
    \chapter{Valutazione del Design}
    In questo capitolo andremo a valutare quello che è il design ottenuto dalle fasi precedenti. La valutazione sarà formata da due step:
    \begin{itemize}
        \item \textbf{L'inspezione:} nella quale andremo a valutare noi stessi in qualità di membri del team il design ottenuto. Useremo la tecnica del \emph{cognitive walkthrougth}, una tecnica che sfrutta l'approccio drammaturgico narrativo.
        \item  \textbf{Il testing:} nella quale verra chiesto ad alcuni utenti di utilizzare i nostri wireframe per valutare l'usabilità del sito. 
    \end{itemize}

    \section{Inspezione}
    Per quanto riguarda l'inspezione stileremo dapprima una lista di task che vogliamo analizzare.
    \begin{enumerate}
        % SIMONE I PRIMI TRE - LUCA GLI ULTIMI 3
        \item Accedere al profilo
        \item Giocare al gioco "Subway surf", metterlo nei preferiti e lasciare un commento. Eliminare poi il commento.
        \item Cercare un utente e aggiungerlo alla lista degli amici.
        \item Modificare la propria foto profilo, caricandone una personalizzata.
        \item Eliminare un gioco dai preferiti.
        \item Impostare il parental control con un limite di due ore di utilizzo ed un blocco per i giochi che superino i 16 anni.
        \item Aggiungere al parental control il blocco per i giochi violenti.
        \item Disabilitare il parental control.
        \item Creare un nuovo account.
    \end{enumerate}
    \subsection{Cognitive Walkthrought}

    \subsection{Valutazione Euristiche (opzionale)}
    \section{User Testing}
    \subsection{Protocollo di testing}
    \subsection{Testing}
    \subsection{Analisi dei risultati}

    \chapter{Conclusioni}
\end{document}