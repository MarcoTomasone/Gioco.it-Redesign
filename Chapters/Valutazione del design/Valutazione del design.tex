\documentclass[../Report.tex]{subfiles}
\usepackage[italian]{babel}
\graphicspath{ {../../Images/} }

\begin{document}
    \chapter{Valutazione del Design}
    In questo capitolo andremo a valutare quello che è il design ottenuto dalle fasi precedenti. La valutazione sarà formata da due step:
    \begin{itemize}
        \item \textbf{L'inspezione:} nella quale andremo a valutare noi stessi in qualità di membri del team il design ottenuto. Useremo la tecnica del \emph{cognitive walkthrougth}, una tecnica che sfrutta l'approccio drammaturgico narrativo.
        \item  \textbf{Il testing:} nella quale verra chiesto ad alcuni utenti di utilizzare i nostri wireframe per valutare l'usabilità del sito. 
    \end{itemize}

    \section{Inspezione}
    Per quanto riguarda l'inspezione stileremo dapprima una lista di task che vogliamo analizzare.
    \begin{enumerate}
        % SIMONE I PRIMI TRE - LUCA GLI ULTIMI 3
        \item Accedere al profilo
        \item Giocare al gioco "Subway surf", metterlo nei preferiti e lasciare un commento. Eliminare poi il commento.
        \item Cercare un utente e aggiungerlo alla lista degli amici.
        \item Modificare la propria foto profilo, caricandone una personalizzata.
        \item Eliminare un gioco dai preferiti.
        \item Impostare il parental control con un limite di due ore di utilizzo ed un blocco per i giochi che superino i 16 anni.
        \item Aggiungere al parental control il blocco per i giochi violenti.
        \item Disabilitare il parental control.
        \item Creare un nuovo account.
    \end{enumerate}
    \subsection{Cognitive Walkthrought}
    \subsubsection{Cognitive walkthrough 1:}
    \textbf{Task:}Accedere al profilo\\
    \textbf{User:} Una donna di 40 anni come Simona
    \subsubsection{Happy path}
    \begin{enumerate}
        \item Cliccare sull'icona dell'utente in alto a destra
        \item Cliccare su accedi con facebook o google qualora si volesse effettuare l'accesso tramite social network oppure inserire username e password negli appositi campi
        \item Cliccare su 'Accedi'
    \end{enumerate}
    \textbf{Walkthrough}:\\
    Simona si trova nell'Home page del sito, individua il pulsante in alto con l'icona del profilo, la clicca e viene reindirizzata alla pagina di login.
    Inserisce l'username e la password ed in seguito ad un errore di battitura non digita la password corretta ed é costretto a modificare la password appena inserita e ripremere il pulsante 'Accedi'

    \textbf{Valutazione:}\\Storia credibile e segue perfettamente l'happy path ma deve comparire un messaggio di errore in caso la password sia errata per un errore di battitura\\

    \textbf{Miglioramenti:}Inserire un messaggio di errore in caso la password sia errata e il pulsante 'Recupera password'\\

    \subsubsection{Cognitive walkthrough 2:}
    \textbf{Task:}Giocare al gioco "Subway surf", metterlo nei preferiti e lasciare un commento. Eliminare poi il commento.\\
    \textbf{User:} Un bambino dell'etá di 12 anni come Mirko
    \subsubsection{Happy Path}
\begin{enumerate}
    \item Cliccare sull'icona di ricerca in alto a destra
    \item Digitare nel campo apposito 'Subway Surf'
    \item Cliccare su 'Invio'
    \item Cliccare sull'immagine riferita al gioco
    \item Cliccare il tastino del cuore ed eseguire l'accesso in caso non sia stato fatto
    \item Scorrere in fondo alla pagina
    \item Digitare il commento nella sezione apposta
    \item Cliccare sul pulsante con l'icona 'Invio'
\end{enumerate}
\textbf{Walkthrough}:\\
    Mirko si trova nell'home page la scorre cercando il gioco di suo interesse e non lo trova.
    Di conseguenza clicca sul pulsante di ricerca e inizia a digitare 'Subway surf' fino che non compare come risultato.
    Una volta trovato lo clicca e viene reindirizzato alla pagina di suo interesse.
    Gioca per un paio di minuti al gioco e clicca sul cuore in alto a destra e clicca sul pulsante dei commenti sotto al cuore che lo reindirizza alla sezione dei commenti in fondo alla pagina.
    Nel campo di testo, digita un commento e successivamente preme invio per pubblicare il commento

    \textbf{Valutazione:}\\Storia abbastanza realistica e abbastanza semplice non riesce tuttavia a riprodurre il gioco a schermo intero.\\

    \textbf{Miglioramenti:}\\Inserire un pulsante per ingradire la finestra del gioco

    \subsubsection{Cognitive walkthrough 3:}
    \textbf{Task:}Cercare un utente e aggiungerlo alla lista degli amici.\\
    \textbf{User:} Un bambino dell'etá di 8 anni come Gaetano
    \subsubsection{Happy Path}
    \begin{enumerate}
        \item Eseguire l'accesso qualora non sia stato fatto 
        \item Cliccare sull'immagine del profilo in alto a destra
        \item Selezionare nel menu a tendina la voce 'Amici'
        \item Qualora non fosse giá selezionato, selezionare la tab 'I miei amici'
        \item Cliccare il pulsante con l'icona dell'utente e un '+'
        \item Digitare il nome dell'amico nel campo
        \item Mandargli la richiesta di amicizia con il tasto '+'
    \end{enumerate}
    \textbf{Walkthrough}:\\
    Una volta eseguito l'accesso va nel menu a tendina riferita all'immagine del profilo dell'utente.
    Seleziona la voce 'Amici' e clicca sul pulsante 'Ricerca' digita il nome dell'utente ma non lo trova chiude quindi quella finestra e clicca sull'icona affianco con l'omino e il '+' da quindi cé un'altra barra di ricerca.
    Digita il nome dell'amico, lo trova e lo aggiunge alla lista degli amici\\

    \textbf{Valutazione:}\\Storia non troppo realista poiché il ragazzo ha solo 8 anni e non é detto che sappia muoversi cosi bene su un sito se non con l'ausilio di un adulto.
    Nonostante il path é errato poiché per prima cosa l'utente cerca nella sua lista di amici e non trova il risultato sperato quindi é costretto a tornare indietro e trovare il pulsante 'aggiungi amici'\\

    \textbf{Miglioramenti:} Mettere in risalto il fatto che la prima ricerca si basa solo sugli amici gia nella lista e non dei nuovi

    \subsubsection{Cognitive walkthrough 7:}
    \textbf{Task:} Aggiungere al parental control il blocco per i giochi violenti.\\
    \textbf{User:} Un padre di 45 anni come Claudio (vedere \ref{section: personas}).\\

    \subsubsection{Happy path}
    \begin{enumerate}
        \item Eseguire l'accesso qualora non sia stato fatto 
        \item Cliccare sul pulsante con l’omino in alto a destra sulla HomePage
        \item Dal menu che si apre scegliere Parental Control
        \item Inserire la password per accedere al Parental Control        \item Nella nuova schermata che si presenta, cliccare sul pulsante "+" al box "Vuoi limitare l'utilizzo di alcuni giochi?"
        \item Dalla finestra di dialogo cercare la categoria violenza (se non è già presente)
        \item Premere il pulsante "+" di fianco la categoria "Violenza"
    \end{enumerate}

    \subsubsection{Walkthrough:}
    Claudio è sulla HomePage del sito Giochi.it e comincia ad esplorarlo. Individua in alto a destra il pulsante del profilo e ci clicca sopra e a dal menu che viene fuori clicca sulla sezione Parental Control. Successivamente, all'apertura della nuova pagina, individua l'input text da inserire e inserisce inizialmente una password errata, poi si corregge e inserisce quella giusta. Dopo essersi aperta la schermata del Parental Control, Claudio, individua il box di limite dei giochi, clicca sul pulsante "+" e nella finestra modale che compare individua subito la categoria Violenza e la seleziona con il pulsante "+" e infine salva.  
    \subsubsection{Valutazione:}
    Storia realistica e segue perfettamente l’happy path, nonostante Claudio abbia sbagliato inizialmente la password è riuscito a completare il task. 

    \subsubsection{Miglioramenti:}
    Inserire un messaggio di errore quando la password inserita è sbagliata.

    \subsubsection{Cognitive walkthrough 8:}
    \textbf{Task:} Disabilitare il Parental Control.\\
    \textbf{User:} Una ragazza di 35 anni come Silvia (vedere \ref{section: personas}).\\

    \subsubsection{Happy path}
    \begin{enumerate}
        \item Eseguire l'accesso qualora non sia stato fatto 
        \item Cliccare sul pulsante con l’omino in alto a destra sulla HomePage
        \item Dal menu che si apre scegliere Parental Control
        \item Inserire la password per accedere al Parental Control 
        \item Nella nuova schermata che si presenta, cliccare sul bottone radio "No", alla domanda "Vuoi attivare il parental control?"
    \end{enumerate}

    \subsubsection{Walkthrough:}
    Silvia vuola disattivare il Parental Control, usato per i suoi alunni, per giocare. Inizialmente è sulla HomePage del sito Giochi.it e comincia trova in alto a destra il pulsante del profilo e ci clicca sopra e dal menu che viene fuori clicca sulla sezione del suo profilo. Successivamente, all'apertura della nuova pagina, individua il pulsante con lo scudo del Parental Control e lo clicca e inserisce la password in modo corretto nella text box per accedervi. Dopo essersi aperta la schermata del Parental Control, Silvia, individua il bottone "No" per disabilitare le restrizioni e torna alla HomePage e chiude la finestra. Riaccedendovi qualche ora giorno dopo si rende conto che il Parental Control era ancora attivo, allora si ricordò di non aver salvato. Silvia rifece tutto da capo ricordandosi di salvare e completò il task.  
    \subsubsection{Valutazione:}
    La storia risulta credibile. L'utente non ha seguito l'happy path ma ha raggiunto l'obiettivo con solamente un passaggio in più. Questo è ottimo perché l'utente può arrivare all'obiettivo in più modi ma allo stesso tempo potrebbe risultare ridondante. Inoltre non è stato inserito nessun messaggio "reminder" per salvare prima di uscire.

    \subsubsection{Miglioramenti:}
    Inserire un reminder per salvare se viene cambiata pagina senza salvare le modifiche.
    \subsubsection{Cognitive walkthrough 9:}
    \textbf{Task:} Creare un nuovo account.\\
    \textbf{User:} Un uomo dai 40 ai 50 anni come Claudio (vedere \ref{section: personas}).\\

    \subsubsection{Happy path}
    \begin{enumerate}
        \item Se è già stato eseguito l'accesso disconnettersi dal profilo
        \item Cliccare sul pulsante con l’omino in alto a destra sulla HomePage
        \item Nella nuova schermata andate nella sezione "Registrati" 
        \item Compila il form co tutti i tuoi dati
        \item Imposta il Parental Control con password "101010"
        \item Conferma di non essere un robot
        \item Premi sul pulsante "Registrati"
    \end{enumerate}

    \subsubsection{Walkthrough:}
    Una volta fatto il logout, Claudio, preme sul pulsante dell'omino nella HomePage.
    Alla schermata che appare si sposta nella sezione "Registrati" e comincia a compilare il form. Dopo aver inserito tutti i dati imposta il Parental Control con la password designata, cerca di dimostrare di non essere un robot e dopo vari tentavi ci riesce (non è molto bravo a distinguere le bici dalle moto). Infine clicca sul pulsante "Registrati" e riesce a completare il task.
    \subsubsection{Valutazione:}

    \subsubsection{Miglioramenti:}

    
    \subsection{Valutazione Euristiche (opzionale)}
    \section{User Testing}
    \subsection{Protocollo di testing}
    \subsection{Testing}
    \subsection{Analisi dei risultati}

    \chapter{Conclusioni}
\end{document}