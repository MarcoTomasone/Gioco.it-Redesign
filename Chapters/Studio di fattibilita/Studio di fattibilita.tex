\documentclass[../Report.tex]{subfiles}
\usepackage[italian]{babel}
\graphicspath{ {../../Images/} }

\begin{document}
    \chapter{Studio di fattibilità}
    \section{Contesto di utilizzo}
    Questa sezione mira ad identificare gli utenti della piattaforma e i loro compiti, in particolare enfatizzando la relazione tra i loro obiettivi (user goal) e le mansioni svolte (task). Inoltre, vengono individuati vincoli ambientali e tecnici rilevanti per le successive fasi di design.
        
        \subsection{Utenti previsti}
        Ci aspettiamo che gli utenti destinatari, che andranno ad utilizzare il sito Gioco.it, sono bambini che vogliono giocare ai vari mini giochi contenuti in esso ma in modo controllato rispettando il concetto di "different twins". Quindi, in questo caso il responsabile (i genitori/tutori) è diverso dall'utente (bambino).\\
        Abbiamo individuato due segmenti target di riferimento, in sintesi:
        \begin{itemize}
            \item Bambini di età compresa fra i 3 e i 18 anni che abbiano un minimo di dimestichezza con il dispositivo e con la navigazione internet. Interessati a giocare ai vari mini giochi presenti sul sito.
            \item Uomini/donne di età compresa fra i 30 e i 50 anni, il cui reddito è medio-basso (tutti ormai abbiamo almeno un dispositivo con navigazione internet) e che abbiano una buona dimestichezza con il dispositivo e con la navigazione internet. Interessati a far giocare il bambino a giochi prevalentemente educativi e bloccando quelli che ritengono inopportuni per l'età del bambino.
        \end{itemize} 

        \subsection{Compiti previsti}
        Durante le interviste sono emerse una serie di “idee per compiti” ricorrenti. Elaboriamo queste idee per produrre un elenco dei compiti più plausibili che il nostro sistema supporterà:
        \begin{itemize}
            \item Creazione di un profilo
            \item Login e logout dal sistema (anche con google o facebook)
            \item Visualizzare i giochi in tendenza
            \item Visualizzare i giochi in base a varie categorie e sottocategorie
            \item Giocare ai vari mini giochi
            \item Ricercare i giochi per parola chiave
            \item Utilizzo di un parental control
            \item Commentare i giochi
            \item Aggiungere i giochi ai preferiti
            \item Aggiungere amici
            \item ecc...
        \end{itemize}

        \subsection{Vincoli ambientali e tecnici}
        \begin{itemize}
            \item \textbf{Vincoli tecnici:} trattandosi di una piattaforma web, è necessario che l'utente disponga di una connessione internet. Quindi supponiamo che tutti i nostri utenti possiedano almeno un personal computer (desktop o laptop) o smartphone con una connessione ad internet.
            \item \textbf{Vincoli culturali:} il sito è solo in lingua italiana e non si può accedere da un altro paese.
            \item \textbf{Vincoli ambientali:} la piattaforma è pensata per essere utilizzata nel tempo libero, pertanto l'ambiente di utilizzo più comune sarà la propria abitazione o un ambiente altrettanto confortevole e accogliente, senza alcun vincolo di tempo.
        \end{itemize}
        Ci aspettiamo quindi, che la maggior parte degli utenti utilizzi il sistema a proprio piacimento, in modo rilassato, esplorativo e possibilmente senza obiettivi.

    \section{Personas}
    Partendo dai nostri target segments, presentiamo quattro personas plausibili, due per ciascun target segments. Partiamo con il protagonista (Gaetano) e poi forniamo altri tre personaggi secondari (Claudio, Mirko e Simona).\\

    \begin{table}[H]
        \begin{tabular}{|c|l|p{7cm}|}
            \hline
            \multirow{5}{*}{\includegraphics[width=5cm, height=5cm]{Gaetano.jpg}} 
                & \textbf{Nome:} & Gaetano Esposito\\ \cmidrule{2-3}
            & \textbf{Età:} & 6 \\ \cmidrule{2-3}
            & \textbf{Stato civile:} & Nubile \\ \cmidrule{2-3}
            & \textbf{Occupazione:} & \makecell{Studente \\ Mantenimento da parte dei genitori} \\ \cmidrule{2-3}
            & \textbf{Abilità tecinche:} &  È molto abile nel riconoscere i dinosauri ed ha imparato ad usare il computer grazie al padre.\\
            \hline
        \end{tabular}
    \end{table}

    \subsubsection{Descrizione}
    Gaetano è un bimbo di 6 anni, l'unico figlio di mamma Jennifer, parruchiera, e papà Claudio, Medico. È un bambino molto perspicace ed attento, ama la natura e gli animali e infatti per il suo quinto compleanno il padre gli ha regalato un bellissimo Golden Retriver di nome Pippo. La sua più grande passione sono i dinosauri, infatti riesce a riconoscerne molti grazie ai libri di storia.
    Gaeteano va due volte a settimana a seguire corsi di piscina, spinto dalla passione della madre, ex nuotatrice amatoriale.\\
    Il padre vorrebbe che imparasse giocando, ed è proprio per questo che gli permette di utilizzare il suo computer fisso per giocare a giochi educativi.

    \vspace{1.5cm}

    \begin{table}[H]
        \begin{tabular}{|c|l|p{7cm}|}
            \hline
            \multirow{5}{*}{\includegraphics[width=5cm, height=5cm]{Claudio.jpg}} 
                & \textbf{Nome:} & Claudio Esposito\\ \cmidrule{2-3}
            & \textbf{Età:} & 45 \\ \cmidrule{2-3}
            & \textbf{Stato civile:} & Sposato \\ \cmidrule{2-3}
            & \textbf{Occupazione:} & \makecell{Medico di base \\ 70.000€ reddito netto annuo} \\ \cmidrule{2-3}
            & \textbf{Abilità tecinche:} &  Conosciuto in tutta la città per la sua professionalità e competenza. Ha un MacBook Pro che utilizza per lavorare e un computer fisso a casa per vedere film e seguire corsi di aggiornamento.\\
            \hline
        \end{tabular}
    \end{table}

    \subsubsection{Descrizione}
    Claudio è un medico di 45 anni, sposato con una bellissima parrucchiera di nome Jennifer e hanno un bimbo di 6 anni, Gaetano, molto intelligente e appassionato di dinosauri e della natura. È uno dei più bravi medici di base della sua città e lo conoscono tutti con il soprannome di Ciruzzo, per le sue origini partenopee. Oltre al suo lavoro, gli piace cimentarsi nell'informatica e utilizza il suo computer fisso per seguire corsi di aggiornamento e guardare film e serie tv. Una delle sue serie tv preferite è l'anime Naruto, infatti lui e Gaetano passano ore davanti al pc a vedere le puntate.\\
    Claudio ci tiene molto all'educazione del figlio, ed è per questo che vorrebbe far giocare al figlio a dei giochi principalmente educativi, e date le passioni di Gaetano, giochi con animali sarebbero ottimi.

    \vspace{1.5cm}

    \begin{table}[H]
        \begin{tabular}{|c|l|p{7cm}|}
            \hline
            \multirow{5}{*}{\includegraphics[width=5cm, height=5cm]{Mirko.jpg}} 
                & \textbf{Nome:} & Mirko Rossi\\ \cmidrule{2-3}
            & \textbf{Età:} & 10 \\ \cmidrule{2-3}
            & \textbf{Stato civile:} & Nubile \\ \cmidrule{2-3}
            & \textbf{Occupazione:} & \makecell{Studente \\ Mantenimento da parte dei genitori} \\ \cmidrule{2-3}
            & \textbf{Abilità tecinche:} & Usa il laptop della madre per fare ricerche scolastiche o per giocare. Gioca spesso con gli amici ed è imbattibile nei giochi d'azione, principalmente gli sparatutto. \\
            \hline
        \end{tabular}
    \end{table}

    \subsubsection{Descrizione}
    Mirko è un ragazzino di 10 anni che frequenta la quinta elementare nella scuola del suo paese. Ha una sorella di 15 anni ed è appassionata di moda. La madre, Susanna, lavora come commercialista in uno studio del loro piccolo borgo cittadino. Il padre, Antonio, esercita la professione di elettricista e spesso è fuori casa. Dopo aver fatto i compiti, Mirko, utilizza il laptop della madre per giocare a diversi videogiochi. La madre, essendo molto apprensiva nei confronti del figlio, tiene molto alla sua educazione, pertanto vorrebbe una maggiore sicurezza sugli usi del suo laptot.

    \vspace{1.5cm}

    \begin{table}[H]
        \begin{tabular}{|c|l|p{7cm}|}
            \hline
            \multirow{5}{*}{\includegraphics[width=5cm, height=5cm]{Simona.jpg}} 
                & \textbf{Nome:} & Simona Tomasone\\ \cmidrule{2-3}
            & \textbf{Età:} & 30 \\ \cmidrule{2-3}
            & \textbf{Stato civile:} & Nubile \\ \cmidrule{2-3}
            & \textbf{Occupazione:} & \makecell{Insegnante di scuola primaria \\ 17.000€ reddito netto annuo} \\ \cmidrule{2-3}
            & \textbf{Abilità tecinche:} &  Ama innovare il modo di insegnare. Abile con il pennello, utilizza il suo laptop per comprare quadri all'asta e per seguire nuovi corsi sui metodi di insegnamento.\\
            \hline
        \end{tabular}
    \end{table}

    \subsubsection{Descrizione}
    Simona è un'insegnate della scuola primaria di 30 anni. È l'unica figlia di una cassiera in pensione di 68 anni e di un operaio di 62 anni, con i quali vive tuttora. Si è diplomata al liceo classico ed in seguito si è laureata in Scienze della formazione all'Università di Bologna.\\
    Ama stare con i bambini, infatti ora insegna nella scuola primaria del suo paese. I suoi hobby principali sono la pittura e leggere romanzi, anche se passa svariate ore su Instagram. Vorrebbe proporre al preside della sua scuola di poter far giocare i suoi alunni ad alcuni giochi educativi su internet, ma la sua paura è che potrebbero finire a giocare anche a giochi non adatti alla loro età.

    \section{Scenari}
    Procediamo ora a fornire uno scenario di utilizzo plausibile per ciascuno dei nostri personaggi, nella speranza di comprendere meglio la relazione tra obiettivi personali e compiti coinvolti.

    \subsection{Basta sparare (Mirko)}
    In questi giorni la scuola di Mirko è chiusa per le vacanze di Natale, quindi ha più tempo libero per dedicarsi ai suoi videogiochi preferiti sul sito Gioco.it, tra cui Subway Clash 2. Questo gioco, per l'età di Mirko, è abbastanza cruento e poco educativo, in effetti venne a conoscenza del sito e del gioco tramite suo cugino Daniele di 15 anni. Susanna, la madre di Mirko, venendo poi a conoscenza delle attività ricreative di Mirko online (Gioco.it), ha capito di intervenire cercando di bloccare l'accesso a questi giochi violenti. Decise così di accedere al sito, creare un profilo a Mirko (con i propri dati) e tramite l'utilizzo del parental control, in modo intuitivo, è riuscita a filtrare solo i giochi per ragazzi dai 10 anni in giù.

    \subsection{Tutto in uno (Simona)}
    Simona ha sentito parlare di Gioco.it da un suo amico che lo usa con costanza e lo ha già usato una o due volte, per curiosità e noia. Ha appena giocato a "A caccia di parole", un gioco educativo dove l'obiettivo è trovare tutte le parole nascoste all'interno di un puzzle.\\
    Un giorno ha deciso di proporre al preside della scuola dove insegna, di poter far giocare i suoi alunni con dei giochi educativi nel laboratorio d'informatica e lui ha accettato subito. Il primo giorno in cui andarono al laboratorio, Simona, dopo che essi cominciarono a giocare, si rese conto che alcuni alunni giocavano a diversi giochi violenti. Lei prese subito in mano la situazione, fece accedere ad ogni alunno con il suo profilo personale e, data la grafica molto semplice ed intuitiva, riuscì a settare il filtro età. Alla fine dell'ora i ragazzi si divertirono molto e furono anche molto più bravi a ricercare le parole nel gioco "A caccia di parole".

    \subsection{Addio ricerca di storia (Gaetano)}
    Gaetano è capitato sul sito Gioco.it navigando su internet per un banner pubblicitario durante una ricerca scolastica, infatti ha lasciato perdere la ricerca e ha passato tutto il pomeriggio a giocare con i vari mini giochi sul sito.\\
    Il giorno dopo è tornato a casa con una nota della maestra per non aver svolto la ricerca di storia. Il padre, Claudio, si è infuriato e ha deciso di controllare le attività di suo figlio al computer di casa. Ha così deciso di aprire il sito Gioco.it, è riuscito i modo semplice a creare un profilo per Gaetano ed inserire i limiti di età per i diversi giochi e anche un limite di tempo per usufruire di essi. Ora Gaetano ha capito che non deve passare troppo tempo davanti ai videogiochi e soprattutto che sono molto più costruttivi i giochi educativi.

    \subsection{Il controllo non è mai troppo (Claudio)}
    Claudio, tornando a casa dal lavoro, ha sentito parlare di Gioco.it dal padre di un amico del figlio, Marco, dicendogli che è stato proprio Gaetano a fargli conoscere il sito e che oramai Marco passa le giornate a giocare agli sparattutto. Claudio, tornando a casa, controlla la cronologia del computer fisso di casa e vede che negli ultimi giorni Gaetano ha passato troppo tempo a giocare ai videogiochi, principalmente agli sparattutto. Allora Claudio decide di creare un profilo a Gaetano ed applicare i modo semplice e veloce delle restrizioni per non farlo giocare a troppi giochi violenti. Decide anche di invitare, tramite il sito, il suo amico Marco in modo da potersi confrontare con classifiche e commenti sui vari videogiochi e anche per fare tenere sotto controllo Marco da suo padre.

\end{document}