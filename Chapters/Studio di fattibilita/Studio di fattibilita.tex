\documentclass[../Report.tex]{subfiles}
\usepackage[italian]{babel}

\begin{document}
    \chapter{Studio di fattibilità}
    \section{Contesto di utilizzo}
        Ora definiamo chi sono gli utenti previsti e quali sono i compiti, quali sono i vincoli tecnici e quelli ambientali.\\
        
        \subsection{Utenti previsti}
        Ci aspettiamo che gli utenti destinatari, che andranno ad utilizzare il sito Gioco.it, sono dei bambini che vogliono giocare ai vari mini giochi contenuti sul sito ma in modo controllato, attraverso un parental control, rispettando il concetto di "different twins". Quindi in questo caso il responsabile (i genitori/tutori) è diverso dall'utente (bambino). Da completare...

        \subsection{Compiti previsti}
        Durante le interviste sono emerse una serie di “idee per compiti” ricorrenti. Elaboriamo su questi idee per produrre un elenco dei compiti più plausibili che il nostro sistema supporterà:
        \begin{itemize}
            \item Creazione di un profilo
            \item Login e logout dal sistema
            \item Visualizzare i giochi in tendenza
            \item Visualizzare i giochi in base a varie categorie
        \end{itemize}

        \subsection{Vincoli ambientali e tecnici}
        Il nostro target è costituito da persone che posso permettersi un cazzo di computer di merda, quindi supponiamo che tutti i nostri utenti possiedano almeno un personal computer (desktop o laptop) o smartphone con una connessione ad Internet.
        Assumiamo infine che la maggior parte, se non la totalità, dei nostri utenti target utilizzerà il nostro sistema nel tempo libero, a casa o in un ambiente altrettanto confortevole e accogliente, senza alcun vincolo di tempo.\\
        Ci aspettiamo anche che la maggior parte degli utenti utilizzi il sistema a proprio piacimento, in modo rilassato, esplorativo e possibilmente senza obiettivi.

    \section{Scenari}
    \section{Personas}
\end{document}