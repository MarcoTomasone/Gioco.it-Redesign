\documentclass[../Report.tex]{subfiles}
\usepackage[italian]{babel}

\begin{document}
    \chapter{Proposta di design}
    Ora procediamo con la fase di design del sito Gioco.it.
    \subsubsection{Analisi degli utenti}
    \textbf{Che target di utenti è atteso?} Come è descritto durante la ricerca etnografica \ref{} ci aspettiamo principalmente due macro target di utenti:
    \begin{itemize}
        \item Bambini e ragazzi come utilizzatori del sito (giocatori)
        \item Adulti come supervisori degli utilizzatori 
    \end{itemize}

    Il sito dovrà quindi prendere in carico le differenti richieste di entrambi i soggetti: la voglia di divertirsi dei giocatori e la propensione all'educazione e alla sicurezza da parte dei supervisori.\\
    \\
    \textbf{Perché gli utenti dovrebbero utilizzare il nostro servizio?} Perché il sito offre la più grande varietà e collezione di giochi presenti sul web e poiché sarà l'unico che offrirà un'attenzione particolare ai bisogni dei supervisori dei giocatori che giocano al sito.

    \subsubsection{Navigazione}
    \textbf{Come migliorare la navigazione tra le categorie?} Nella vecchia versione di Gioco.it alcune macro categorie non erano visibili nel menù principale e allo stesso modo c'era una grande discordanza tra le sottocategorie mostrate nella home page e quelle mostrate nel menù delle categorie e nelle categorie consigliate, sarà quindi necessario un grande lavoro di coerenza e uniformazione dei contenuti per far sì che gli utenti possano navigare correttamente nel sito.

    \subsubsection{Etica del sito}
    \textbf{I supervisori devono ancora essere preoccupati?} Al momento il sito Gioco.it non fornisce nessun tipo di informazioni ai supervisori dei giocatori, inoltre non è possibile impostare un parental control e/o dei limiti di utilizzo del sito (es: tempo di utilizzo). Con la nuova versione del sito i supervisori possono essere molto più tranquilli nel lasciare i giocatori liberi di giocare sapendo che il sito Gioco.it è un luogo sicuro.

    \section{Architettura dell'informazione}
    Valutando quello che è l'attuale architetture dell'informazione del sito Gioco.it, abbiamo deciso di utilizzare un approccio Top-Down mantenendo quindi quello che è il workflow di navigazione del sito. Elencheremo di seguito l'architettura del sito progettata a partire da quella originale che terrà conto, ovviamente, dei problemi riscontrati nelle fasi Ispezione e Testing \ref{}:
    \begin{enumerate}
        \item \textbf{Home page:} la home page deve essere capace di offrire un'overview generale del sito, mostrando tutte quelle che sono le funzionalità a disposizione degli utenti. Deve soprattutto offrire una buona rappresentazione dei giochi a disposizione per i nuovi utenti mentre deve semplificare l'utilizzo, fornendo scorciatoie per i giochi più frequenti, agli utenti che tornano sul sito. Inoltre, la grafica deve essere semplificata mantenendo un design più minimalista e meno ricco.
        \item \textbf{Barra di ricerca:} barra che permette di ricercare dei giochi per nome. Dovrebbe però permettere anche una minima tolleranza agli errori.
        \item \textbf{Schermata del gioco:} essa deve offrire all'utente la possibilità di mettere il gioco nei preferiti, commentare il gioco, leggerne la descrizione, ricevere un aiuto per giocare e giocare. È importante ricordare che deve proporre dei giochi simili nel caso in cui il giocatore voglia provare nuove esperienze.
        \item \textbf{Menù delle categorie:} il nuovo menù delle categorie dovrà avere coerenza e mostrare tutte le categorie di giochi presenti sul sito senza nascondere alcune categorie (come accade con il design attuale). Riteniamo necessario anche una semplificazione delle categorie e sottocategorie attualmente presenti.
        \item \textbf{Card del gioco:} attualmente in ogni parte del sito, le card di ogni gioco contengono solo due elementi: immagine e nome del gioco. Come emerso dalle interviste \ref{}, i supervisori preferiscono avere delle informazioni aggiuntive che supportino la scelta del gioco. È necessario quindi aggiungere informazioni relative al tipo di gioco, ad eventuali limitazione di età proprie del gioco.
        \item \textbf{Lista di giochi per categoria:} è la sezione in cui gli utenti visualizzano tutte le card dei giochi appartenenti a quella categoria selezionata e dovrà contenere la possibilità di filtrare e ordinare i giochi in base a determinate caratteristiche.
        \item \textbf{Account:} sarà composto da varie sotto sezioni
        \begin{itemize}
            \item \textbf{Schermata di login e registrazione:} offre la possibilità di eseguire l'accesso al proprio account o di registrarsi (anche tramite google o facebook).
            \item \textbf{Giochi preferiti:} permette di visualizzare i giochi salvati nel proprio account ed eliminarli dalla lista dei giochi preferiti.
            \item \textbf{Amici:} permette di visualizzare la lista di amici e le richieste di amicizia inviate e ricevute.
            \item \textbf{Immagine del profilo:} permette di visualizzare e modificare l'immagine del proprio profilo visibile agli altri utenti, con la possibilità di caricare un'immagine a proprio piacimento.
            \item \textbf{Impostazioni:} permette di modificare le impostazioni del proprio account come aggiornare la password e modificare le informazioni personali.
            \item \textbf{Parental control:} offre la possibilità, al supervisore, di impostare dei filtri (es: categorie) e delle limitazioni (es: tempo ed età) al giocatore. Inoltre, l'accesso a questa sezione è limitato ai supervisori.
        \end{itemize}
        \item \textbf{Assistenza:} offre servizi di aiuto e assistenza tramite le FAQ.
    \end{enumerate}

\end{document}